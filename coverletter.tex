%!TEX TS-program = xelatex
%!TEX encoding = UTF-8 Unicode
% Awesome CV LaTeX Template for Cover Letter
%
% This template has been downloaded from:
% https://github.com/posquit0/Awesome-CV
%
% Authors:
% Claud D. Park <posquit0.bj@gmail.com>
% Lars Richter <mail@ayeks.de>
%
% Template license:
% CC BY-SA 4.0 (https://creativecommons.org/licenses/by-sa/4.0/)
%


%-------------------------------------------------------------------------------
% CONFIGURATIONS
%-------------------------------------------------------------------------------
% A4 paper size by default, use 'letterpaper' for 

%-------------------------------------------------------------------------------
% A4 paper size by default, use 'letterpaper' for US letter
\documentclass[12pt, a4paper]{awesome-cv}

% Configure page margins with geometry
\geometry{left=1.4cm, top=.8cm, right=1.4cm, bottom=1.8cm, footskip=.5cm}

% Specify the location of the included fonts
\fontdir[fonts]


 
% If you would like to change the social information separator from a pipe (|) to something else
\renewcommand{\acvHeaderSocialSep}{\quad\textbar\quad}

\fontdir[fonts]

% \definecolor{lighttext}{HTML}{CA63A8}

% Set false if you don't want to highlight section with awesome color

% If you would like to change the social information separator from a pipe (|) to something else
\renewcommand{\acvHeaderSocialSep}{\quad\textbar\quad}
% Color for highlights
% Awesome Colors: awesome-emerald, awesome-skyblue, awesome-red, awesome-pink, awesome-orange
%                 awesome-nephritis, awesome-concrete, awesome-darknight

\colorlet{awesome}{red}

% Uncomment if you would like to specify your own color
% \definecolor{awesome}{HTML}{CA63A8}

% Colors for text
% Uncomment if you would like to specify your own color
% \definecolor{darktext}{HTML}{2a2d2f}
% \definecolor{text}{HTML}{2a2d2f}
% \definecolor{graytext}{HTML}{2a2d2f}
% \definecolor{lighttext}{HTML}{2a2d2f}

% Set false if you don't want to highlight section with awesome color
% Colors for text
% Uncomment if you would like to specify your own color

%
%
\definecolor{darktext}{HTML}{000000}
\definecolor{text}{HTML}{000000}
\definecolor{graytext}{HTML}{000000}
\definecolor{lighttext}{HTML}{000000}

\setbool{acvSectionColorHighlight}{true}

%\photo[circle,noedge,right]{figs/IMG_20190210_183945-01.jpeg}




%-------------------------------------------------------------------------------
%	PERSONAL INFORMATION
%	Comment any of the lines below if they are not required
%-------------------------------------------------------------------------------
% Available options: circle|rectangle,edge/noedge,left/right

\name{Ismail}{EZZAKI}
\position{Master Student{\enskip\cdotp\enskip}Freelancer Programmer}
\address{812 Ait kdif Ouarzazate Morocco}

\mobile{(+212) 708070221}
\email{ismail.ezzaki@edu.uca.ma}
\homepage{ismailezzaki.me}
\github{ismailezzaki96}
% \gitlab{gitlab-id}
% \stackoverflow{SO-id}{SO-name}
% \twitter{@twit}
% \skype{skype-id}
% \reddit{reddit-id}
% \medium{madium-id}
% \googlescholar{googlescholar-id}{name-to-display}
%% \firstname and \lastname will be used
% \googlescholar{googlescholar-id}{}
% \extrainfo{extra informations}

%\quote{``be the change you want to see in the world"}



%-------------------------------------------------------------------------------
 



%-------------------------------------------------------------------------------
%	LETTER INFORMATION
%	All of the below lines must be filled out
%-------------------------------------------------------------------------------
% The company being applied to

% The title of the letter
\lettertitle{Cover Letter}
% How the letter is opened
\letteropening{I am writing this letter to express my powerful interest in the Ph.D. position in machine learning and particle physics:” unsupervised Deep Learning Tools for the Discovery of Tomorrow” at the Cluster of Excellence Quantum Universe in 2021.}
% How the letter is closed
\letterclosing{I hope that with this letter I informed you sufficiently of my ambitions and why they fit well into this Ph.D. thesis. I will gladly answer questions you may have and hope to hear from you soon.
Sincerely.}
% Any enclosures with the letter


%-------------------------------------------------------------------------------
\begin{document}

% Print the header with above personal informations
% Give optional argument to change alignment(C: center, L: left, R: right)

% Print the footer with 3 arguments(<left>, <center>, <right>)
% Leave any of these blank if they are not needed

% Print the title with above letter informations
\makelettertitle

%-------------------------------------------------------------------------------
%	LETTER CONTENT
%-------------------------------------------------------------------------------
\begin{cvletter}

\lettersection{About Me}
I graduated last year with a Master's degree in "high energy and computational physics" from Cadi Ayyad University in Morocco. Pursuing my Ph.D. with experts and doing research in the latest topics of physics is my childhood dream.
My current research interests are in supervised and unsupervised machine learning in high energy physics.

It took me a long journey during my career before I came to this well-defined area of interest. It started in the second year of my Physics B.Sc. During that time, like every student, I had a vague interest in high energy physics. I did not know yet if this interest was more theoretical or experimental. Nevertheless, After Google's AlphaGo became the first Computer Go program to beat an unhandicapped professional human player, I started learning about Artificial intelligence and applied it in many physics and no physics problems.
\lettersection{Why this thesis?}

I do think that using Unsupervised Machine Learning in BSM Physics Searches is useful because there is an infinite number of models to test, but with unsupervised machine learning, we can train an anomaly detection algorithm on the standard model background. We can then pass in any given event and the algorithm will assign a measure of anomalousness to it. If we assume that the signal is kinematically different from the background, this allows us to create a method of detecting a signal without making any further assumptions about it.

With unsupervised machine learning, we can develop a technique that can differentiate signal from background without the need of making assumptions about the signal. This technique is not enough to discover a signal on its own, but it will provide a powerful tool that can determine signal regions to explore further as the old saying goes:" Shoot for the moon. If you miss it, you will still land among the stars"

I think that Cluster of Excellence Quantum Universe can help me with these ambitions since it has a strong Machine Learning \& Particle Physics group with access to formidable computational resources.
\lettersection{Why Me?}
I would see myself as an ideal candidate for this Ph.D. since, besides my thorough theoretical knowledge obtained during my master in high energy and computational physics, I do also possess strong numerical and data analysis \& deep learning skills obtained during multiple research projects as seen from my CV; I have a lot of experience in programming languages (Python and C++) and frameworks (ROOT, TensorFlow) Besides this, due to working with supervisors, whose specializations did not always directly include numerical simulations, I learned to work on numerical and computational issues independently using coursera \& github.

\end{cvletter}












%-------------------------------------------------------------------------------
% Print the signature and enclosures with above letter informations
\makeletterclosing

\end{document}
