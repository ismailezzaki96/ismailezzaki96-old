%\section{Theorem}
% --------------------------------------------------- Slide --
%\subsection{Theorem Code}
\label{theoremCode}
\begin{frame}\frametitle{Bilinear covariants}
\begin{itemize}
	\item 

All spinor fields live in the a 4-dimensional spacetime \\
spacetime time algebra
\begin{center}
\begin{tabular}{|c|c|c|c|}\hline
\textbf{$\Gamma$} & \textbf{transform as} & \textbf{\# of $\gamma$'s} & \textbf{\# of composants}\\\hline
1 & scalar & 0 & 1 \\\hline
$\gamma^\mu$ & vector & 1 & 4\\\hline
$\sigma \gamma^\nu$ & tensor & 2 & 6\\\hline
$\gamma^5 \gamma^\mu$ & axial vector & 3 & 4\\\hline
$\gamma^5$ & pseudoscalar & 4 & 1\\\hline
\end{tabular}
\end{center}
	\item 
This exhausts all possibilities. The total number of components is 16, meaning that the set $\{ 1-\gamma^\mu-\sigma \gamma^\nu-\gamma^5 \gamma^\mu-\gamma^5\}$
	\item 
Makes a complete basis for any four-by-four
matrix.
	\item 
Such $\overline{\Psi} \Gamma\Psi$ currents are called \textbf{bilinear covariants}.
\end{itemize}
\end{frame}
