

\section{La thermodynamique des trous noirs}
\begin{frame}{\underline{\secname} }

%\begin{block}{La thermodynamique}
%Science qui sait comprendre des aspects essentiels des systèmes macroscopiques sans connaitre leurs aspects microscopiques.
%\end{block}


\begin{block}{L’expérience de pensée de Wheeler (1970)}
Que se passe-t-il si nous jetons une tasse de thé dans un trou noir?
\pause
\begin{enumerate}
		\item Le thé est chaud - il a une entropie $(S_{the} > 0)$
		\pause
	\item Le trou noir absorbe tout et n'a pas de structure interne $(S_{TN} = 0)$
\pause
	\item $(S_{total} = 0)$  $\rightarrow$ violation du deuxième principe de la thermodynamique 
\end{enumerate}	
	
\end{block}

\pause
\textbf{Bekenstein}: Le thé a une masse  \Rightarrow il va augmenter la masse du trou noir \Rightarrow l'aire de l'horizon doit augmenter + Théorème des aires de Hawking

\pause

\begin{center}
	%suggéra donc que l’entropie généralisé ne peut que croı̂tre.
{\Large $A$ (l'aire de l'horizon)   $\Leftrightarrow$     $S$ (l’entropie)}
\end{center}


%The total area of a closed system never decreases.


\end{frame}

\subsection{La  température  et l’entropie }
\begin{frame}{\underline{\secname} : {\small \subsecname}}
\begin{center}
\textbf{	Débat entre Hawking et Beckenstein }
	
\end{center}
{\color{red}Hawking}: l'analogie entre le théorème d'aire et la $2^{\`{e}me}$ loi de la thermodynamique n'est qu'une question de coïncidence.

\vspace{10pt}
{\color{red}Beckenstein}: Je n'en suis pas convaincu. Nulle part dans la nature, la deuxième loi de la thermodynamique n'est violée. Pourquoi les trous noirs seraient-ils une exception ?
\vspace{10pt}

{\color{red}Wheeler} (Le directeur de thèse de Beckenstein) a dit à Beckenstein: Votre idée est tellement folle qu'elle pourrait bien être vraie.
\vspace{10pt}

{\color{red}Hawking}: si un trou noir a une entropie, il doit avoir une température, et s'il a une température, il doit irradier comme un corps noir. Mais si rien ne peut s'échapper d'un trou noir, comment peut-il rayonner ?


\end{frame}

\begin{frame}{\underline{\secname} : {\small \subsecname}}


\begin{center}
	\textbf{1974: fin du débat et découverte du rayonnement Hawking}
\end{center}

\begin{columns}
	\begin{column}{0.6\linewidth}

\begin{itemize}
			 \setlength\itemsep{0.5em}
	\item Les effets quantiques près de l'horizon des événements permettent aux particules de s'échapper du potentiel gravitationnel.

\item Le trou noir perd de l'énergie (et de la masse) 

\item Les trous noirs rayonnent comme un corps noir avec une température
\end{itemize}

\begin{center}
	%suggéra donc que l’entropie généralisé ne peut que croı̂tre.
\textbf{	{\large $\kappa$ (la gravité de surface)   $\Leftrightarrow$     $T$ (la température)}}
\end{center}

\end{column}
\begin{column}{0.4\linewidth}
	
      \begin{figure}
	\centering
	\includegraphics[width=1.2\linewidth,height=170pt]{figures/hawking}
\end{figure}

\end{column}
\end{columns}

\end{frame}
\begin{frame}{\underline{\secname} : {\small \subsecname}}

\begin{block}{l’entropie de Trou Noir}
L'entropie du trou noir est liée à la surface de l'horizon par la relation,

$$S=\frac{k_{B} c^{3}}{4 G \hbar} A$$
\end{block}

\begin{block}{la température de Trou Noir}
La tempurature du trou noir (formule de Bekenstein-Hawking)

$$T_{h}=\frac{\hbar}{2 \pi k_{b} c} \kappa$$
\end{block}

\pause
\textbf{Les formules contient toutes les constantes fondamentales:}\\

%\begin{eqnarray*}\text{
%la constante de Planck &$\hbar$&\\
%la constante de Boltzmann &$k_b$&\\
%la vitesse de la lumière &$c$&\\
%la constante de Newton &$G$&\\
%}
%\end{eqnarray*}  \Rightarrow $$ \text{La théorie d’unification} $$

	$$ \hbar, k_b, c, G,\Rightarrow \text{La théorie d’unification}$$
\end{frame}

%\begin{frame}{\underline{\secname} : {\small \subsecname}}
%\begin{block}{Évaporation des trous noirs}
%Grâce au résultat de Hawking on peut calculer La luminosité et La durée de vie :
%
%\vspace{20pt}
%\begin{columns}
%	\begin{column}{0.5\linewidth}
%\textbf{la luminosité du rayonnement de Hawking }
%\begin{equation*}
%L=A \sigma T^{4}=\frac{\hbar c^{2}}{3840 \pi a^{2}}=\frac{\hbar c^{6}}{15639 \pi G^{2} M^{2}}
%\end{equation*}
%\end{column}
%\begin{column}{0.5\linewidth}
%\textbf{La durée de vie d'un trou noir}
%
%\begin{equation*}
%\tau=\int_{0}^{\tau} \mathrm{d} t=-\frac{15360 \pi G^{2}}{h c^{4}} \int_{0}^{M} M^{2} \mathrm{d} M
%\end{equation*}
%
%\end{column}
%\end{columns}
%
%\end{block}
%\vspace{20pt}
%\pause
%\begin{center}
%	{\Large $\Rightarrow \tau \cong 0 \quad  L \cong 0 $ }
%\end{center}
%
%\end{frame}


\subsection{Les $\rom{4}$ lois de la thermodynamique}
\begin{frame}{\underline{\secname} : {\small \subsecname}}

\begin{block}{Principe Zéro}
	
	La gravité de surface $\kappa$ d'un trou noir stationnaire est constante sur toute la surface de l'horizon	
\end{block}
\begin{block}{Premier Principe}
	$$dM=\dfrac{k}{8\pi}\delta A+\Omega_{h}\delta J+\Phi_{h}\delta Q$$
	la variation de la masse entraîne une variation de l'énergie cinétique angulaire $\Omega_{h}\delta J$, une variation de l'énergie potentielle électrique $\Phi_{h}\delta Q$ et une variation d'énergie de rayonnement $\dfrac{k}{8\pi}\delta A$.
\end{block}

\end{frame}

\begin{frame}{\underline{\secname} : {\small \subsecname}}
\begin{block}{Dexième Principe}
 L'aire A de l'horizon des événements de chaque trou noir ne peut pas décroître $\delta A\geq 0$. 

\vspace{10pt}

\textbf{le deuxième principe généralisée de la thermodynamique :} L'entropie commune à l'extérieur du trou noir plus l'entropie du trou noir ne diminue jamais $S=S_{T N}+S_{\text{ext}}$
$$
\delta S \geq 0
$$


\end{block}
\begin{block}{Troisième Principe}
 On ne peut pas atteindre $\kappa = 0$ par aucun processus.
\end{block}

\end{frame}


\subsection{L’analogie avec la thermodynamique standard}


\begin{frame}{\underline{\secname} : {\small \subsecname}}
\begin{center}
	\textbf{Où est le terme $P \delta V$ dans le premier principe?}
		$$dM=\dfrac{k}{8\pi}\delta A+\Omega_{h}\delta J+\Phi_{h}\delta Q$$
\end{center}
\pause
Solution : 
\begin{itemize}		 \setlength\itemsep{0.5em}
	\item traiter $\Lambda$ comme une pression thermodynamique \item
Masse $M$ interprétée comme une enthalpie plutôt qu'une énergie \item
nous pouvons l'utiliser pour calculer le volume thermodynamique associé au trou noir $$V= \left(\frac{\partial M}{\partial P} \right)_S$$
\end{itemize}
\pause 
le premier principe devient la formule suivante:
 %TODO 
\begin{center}
	$$dM=\dfrac{k}{8\pi}\delta A+\Omega_{h}\delta J+\Phi_{h}\delta Q +P \delta V$$
 
\end{center}

%The PV term in this equation can be though of as the contribution to the mass-energy of the
%black hole due the negative energy density of the vacuum, ǫ = −P, associated with a positive
%cosmological constant. If the black hole has volume V then it contains energy ǫV = −PV
%and so the total energy is U = M − PV.

\end{frame}




\begin{frame}{\underline{\secname} : {\small \subsecname}}
\begin{block}{ L’analogie avec la thermodynamique standard }
		
	\begin{eqnarray*}
	\textrm{Enthalpie}\quad  H &\leftrightarrow&  M\quad  \textrm{Masse}   \nonumber\\
	\textrm{Température}\quad  T &\leftrightarrow&  \frac{\kappa}{2\pi} \quad  \textrm{Gravité de surface}   \nonumber\\
	\textrm{Entropie}\quad  S &\leftrightarrow&  \frac{A}{4 } \quad  \textrm{l'aire de l'horizon}   \nonumber\\
	\textrm{Pression}\quad  P &\leftrightarrow&  -\frac{\Lambda}{8\pi} \quad  \textrm{Constante cosmologique}   \nonumber\\
	\textrm{Volume}\quad  V &\leftrightarrow&  \left(\frac{\partial M}{\partial P} \right)_S \quad  \textrm{Volume thermodynamique}   \nonumber\\
	\end{eqnarray*}

	
\end{block}
\end{frame}

\begin{frame}{\underline{\secname} : {\small \subsecname}}
% Une fois le volume thermodynamique connu 
\begin{block}{La stabilité thermodynamique}

\begin{itemize}
	\item  L'équation d'état du "fluide" 
\begin{equation*}
P=P(V,T,J,Q)
\end{equation*}
\item La thermodynamique d'équilibre est régie par l'énergie libre de Gibbs: {\color{green}\textbf état équilibre correspond au minimum global de G}
\begin{equation*}
G=M-TS=G(P,T,J, Q)\,. 
\end{equation*}
\item La stabilité thermodynamique locale est caractérisée par {\color{green}\textbf la positivité de la chaleur spécifique} à pression constante (et $Q$ où $J$ fixe) 
\begin{equation*}
C_P\equiv C_{P,J,Q}=T\left(\frac{\partial S}{\partial T}\right)_{P,J,Q}
\end{equation*}	

\end{itemize}	
	\end{block}
	

\end{frame}
