\section{Les trous noirs dans la théorie}

\subsection{Quelques mots sur la relativité générale}
%\begin{frame}{\underline{\secname} : {\small \subsecname}}
%
%
%\note{{\Huge cccccccccc}}
%\begin{block}{la relativité restrinte}
%	
%$$
%d s^{2}=-d t^{2}+d x^{2}+d y^{2}+d z^{2}
%$$	
%\end{block}
%  
%
%
%\begin{block}{le principe d’équivalence}
%\textit{"En tout point de l'espace-temps $\xi_{\mu}$, il est possible de choisir un système de coordonnées local $\xi_{\alpha}$ dans lequel les lois de la physique sont les mêmes que dans la relativité restreinte."}	
%	
%\end{block}
%
%
%\begin{block}{la relativité générale }
%la relativité générale = relativité restrinte +  le principe d’équivalence 
%$$
%d s^{2}=g_{\mu \nu} d x^{\mu} d x^{\nu}
%$$	
%	
%	\end{block}
%
%
%\end{frame}
\begin{frame}{\underline{\secname} : {\small \subsecname}}
\begin{block}{\Rightarrow Les équations d’Einstein}
\begin{itemize}
  \setlength\itemsep{0.5em}
\item Des équations dynamiques qui décrit comment la matière et l’énergie modifient la géométrie de l’espace-temps.
\item  L’équation la plus simple possible satisfaisant au principe d’équivalence.
\item Elle redonne l’équation de Newton dans une limite appropriée (la limite non relativiste).
\end{itemize}

\begin{columns}
	\begin{column}{0.5\linewidth}

\begin{eqnarray*}\label{einstien}
R_{\mu\nu}-\dfrac{1}{2}Rg_{\mu\nu}+g_{\mu\nu}\Lambda&=&8\pi GT_{\mu\nu} \\
\text{Courbure} &=&  \text{Matière énergie}
\end{eqnarray*}



\end{column}
\begin{column}{0.5\linewidth}

\begin{figure}
	\centering
	\includegraphics[width=0.9\linewidth,height=0.45\linewidth]{figures/space}
\end{figure}

\end{column}
\end{columns}

\end{block}


\end{frame}
\begin{frame}{\underline{\secname} : {\small \subsecname}}

Il est possible d'obtenir l'équation d'Einstein à partir du principe de moindre action: 
\begin{equation*}	
{\displaystyle S=\int \left[{\frac {1}{2\kappa }}(R-2\Lambda )+{\mathcal {L}}_{\mathrm {M} }\right]{\sqrt {-g}}\,\mathrm {d} ^{4}x}
\end{equation*}
\vspace{20pt}

\pause
\textbf{3 Solutions explicites avec degré maximal de symétrie des équations d'Einstein :}
\begin{itemize}
	  \setlength\itemsep{0.5em}
	\item Si $\Lambda = 0$ \Rightarrow la métrique de Minkowski $g=-dt^2+dx^2+dy^2+dz^2$.
	\item Si $\Lambda  > 0$ \Rightarrow la métrique de l'espace de-Sitter
	\item Si $\Lambda < 0$ \Rightarrow la métrique de l'espace anti-de-Sitter.
\end{itemize}


\end{frame}

\subsection{Les trous noirs dans la  relativité générale }

\begin{frame}{\underline{\secname} : {\small \subsecname}}
\begin{block}{Trou noir de Schwarzschild}
	
	
	\begin{itemize}
		\item Solution de l'équation d'Einstein dans le
		cas d'un champ gravitationnel statique à symétrie sphérique.
		\item La métrique est donnée par :
\begin{equation*}\label{schwarzchild}
d s^{2}=-\left(1-\frac{r_{S}}{r}\right) d t^{2}+\left(1-\frac{r_{S}}{r}\right)^{-1} d r^{2}+r^{2} d \Omega^{2}
\end{equation*}

\begin{equation*}r_{\mathrm{S}}=2 G M
\end{equation*}

			\item La métrique de Schwarzschild est singulière en  $ r = 2GM$ et $ r=0$
			

	\end{itemize}
	
\end{block}
\end{frame}



\begin{frame}{\underline{\secname} : {\small \subsecname}}
\begin{block}{Trou noir de Schwarzschild}
La courbure scalaire donne l’expression suivante :
$$R^{\mu\nu\alpha\beta}R_{\mu\nu\alpha\beta}=\dfrac{12r_{S}^{2}}{r^{6}}$$
\begin{itemize}
		  \setlength\itemsep{0.5em}
	\item $r = 0$ est une vraie singularité
\item $r=2GM$ est une singularité de coordonnées.(Horizon des évènements) \pause $\rightarrow$  \textbf{peut être effacée par un changement de coordonnées}
\end{itemize}

\vspace{20pt}

\begin{block}{	La métrique de Schwarzschild dans les coordonnées de Kruskal-Szekeres}
\begin{equation*}
ds^{2}=\dfrac{4r_{s}^{3}}{r}e^{-\frac{r}{r_{s}}}(du^{2}-dv^{2})+r^{2}(d\theta^{2}+sin^{2}\theta d\phi^{2})
\end{equation*}
\end{block}



\end{block}
\end{frame}

%
%\begin{frame}{\underline{\secname} : {\small \subsecname}}
%\begin{center}
%	\textbf{Les diagrammes conformes:}
%\end{center}
%	
%	  \begin{columns}
%\begin{column}{0.5\linewidth}
%
%Les diagrammes conformes constituent un bon moyen de faire correspondre des espace-temps complexes à un diagramme relativement simple. \\
%
%\vspace{20pt}
%\begin{center}\textbf{
%	l'infini $\infty$ $\rightarrow$ fini}
%\end{center}
%	
%\end{column}
%\begin{column}{0.5\linewidth}
%      \begin{figure}
%	\centering
%	\includegraphics[width=\linewidth]{figures/penrose_schw}
%	\caption{Diagramme de Penrose d'un trou noir de Schwarzschild}
%\end{figure}
%
%
%
%\end{column}
%\end{columns}
%\end{frame}


\begin{frame}{\underline{\secname} : {\small \subsecname}}

\begin{block}{Trou noir de Reissner Nordström}
	
	
	\begin{itemize}
		\item Solution de l'équation d'Einstein en présence de charge $Q$.
		$$R_{\mu\nu}-\dfrac{1}{2}g_{\mu\nu}R+g_{\mu\nu}\Lambda=8G\pi T_{\mu\nu}$$
		avec:
		$$T_{\mu\nu}=\dfrac{1}{4\pi}F_{\mu}^{\delta}F_{\nu\delta}-\dfrac{1}{4}g_{\mu\nu}F_{\alpha\beta}F^{\alpha\beta}$$
		\item La métrique est donnée par :
		$$ds^{2}=-(1-\dfrac{2M}{r}+\dfrac{Q^{2}}{r^{2}})dt^{2}+(1-\dfrac{2M}{r}+\dfrac{Q^{2}}{r^{2}})^{-1}dr^{2}-r^{2}d\theta^{2}-r^{2}sin^{2}(\theta)d\phi^{2}$$ 
	\end{itemize}
	
\end{block}
\end{frame}


\begin{frame}{\underline{\secname} : {\small \subsecname}}
\begin{block}{Trou noir de Reissner Nordström}

  \begin{columns}
	\begin{column}{0.5\linewidth}

\begin{itemize}
	\item 
	Si $G M^{2}<Q^{2}$ : Pas d'horizon des événements, 
	\item 
	Si $G M^{2}>Q^{2}$ : C'est une situation qui peut physiquement résulter d'un effondrement gravitationnel.\\
	Deux horizons $r_{+}$ et $r_{-}$
	\begin{equation*}
	r_{\pm}=G M \pm \sqrt{G^{2} M^{2}-G Q^{2} }
	\end{equation*}
	Et une singularité à $r = 0$
	\item 
	Si $G M^{2}=Q^{2}$ : Trou noir extrémal
\end{itemize}


\end{column}
\begin{column}{0.5\linewidth}
	

\begin{figure}[H]
	\begin{center}
		\includegraphics[width=\textwidth]{figures/plot.png}
	\end{center}
	\caption{La fonction $ f(r)=1-{2 G M}/{r}+{G Q^{2} }/{r^{2}}$ }
	\label{r-n}
\end{figure}
	
\end{column}
\end{columns}


\end{block}
\end{frame}



\begin{frame}{\underline{\secname} : {\small \subsecname}}
\begin{block}{Trou noir de Kerr}
	
	
	\begin{itemize}
		\item Solution exacte des équations d'Einstein permettant de décrire le comportement de l'espace-temps
		autour d'un trou noir en rotation $J\neq 0$ .
		
		\item La métrique est donnée par (avec les coordonnées de Boyer-Lindquist) :
		$$ds^{2}=-(1-\dfrac{2Mr}{\Sigma})dt^{2}+\dfrac{\Sigma}{\Delta}dr^{2}+\Sigma d\theta^{2}+\dfrac{Asin^{2}\theta}{\Sigma} d\phi^{2}-\dfrac{4Marsin^{2}\theta }{\Sigma} dt d\phi$$
		$$A=(r^{2}+a^{2})^{2}-\Delta a^{2}sin^{2}\theta ,\quad
		\Sigma =r^{2}+a^{2}cos^{2}\theta ,\quad
	     \Delta=r^{2}-2Mr+a^{2}  ,\quad
	     a=J / M
	     $$
		\item Si $M > a$  la métrique est singulière en:
		$$r_{\pm} = M \pm\sqrt{M^{2}-a^{2}}$$
		
	\end{itemize}
	
\end{block}
\end{frame}


%\begin{frame}{\underline{\secname} : {\small \subsecname}}
%\begin{block}{Trou noir de Kerr}
%	
%	\begin{columns}
%		\begin{column}{0.5\linewidth}
%\textbf{Ergosphère}:
%\vspace{20pt}
%
%C'est une région délimitée par la limite statique à l'extérieur, et par l'horizon externe à l'intérieur, dans laquelle rien ne peut rester immobile.\\
%\vspace{30pt}
%{\Rightarrow Il est possible d’extraire de l’énergie de rotation d’un trou noir de Kerr : \textbf{ Processus d’extraction d’énergie de Penrose }}
%		
%	\end{column}
%	\begin{column}{0.5\linewidth}
%
%
%\begin{figure}[H]
%	\begin{center}
%		\includegraphics[width=\textwidth]{figures/ergosphere.png}
%	\end{center}
%	\caption{Une vue de côté d'un trou noir de Kerr}
%	\label{kerr}
%\end{figure}
%
%\end{column}
%\end{columns}
%\end{block}
%\end{frame}

\begin{frame}{\underline{\secname} : {\small \subsecname}}

\begin{block}{Trou noir Kerr-Newman}
	
	
	\begin{itemize}
		\item  une solution de l'équation d'Einstein dans le cas d'un trou noir chargé et en rotation .
		
		\item Sa métrique est donnée par :
		$$ds^{2}=-\dfrac{\Delta}{\rho^{2}}(dt-asin^{2}\theta d\phi)^{2}+\dfrac{sin^{2}}{\rho^{2}}[(r^{2}+a^{2})d\phi-adt]^{2}+\dfrac{\rho^{2}}{\Delta}dr^{2}+\rho^{2}d\theta^{2}$$
		$$\Delta=r^{2}-2Mr+a^{2}+Q^{2} ,\quad	\rho^{2}=r^{2}+a^{2}cos^{2}\theta ,\quad a={J}/{M}$$
		
		
	\end{itemize}
\pause
\begin{itemize}
			  \setlength\itemsep{0em}
	\item  $Q=a=0$ \Rightarrow métrique de Schwarzschild
\item $a=0$ \Rightarrow métrique de Reissner Nordstrom
\item $ Q=0$ \Rightarrow métrique de Kerr
\item $M=Q=a=0$ \Rightarrow métrique d’un espace de Minkowski vide
\end{itemize}
\end{block}
\end{frame}


\begin{frame}{\underline{\secname} : {\small \subsecname}}
\begin{block}{The no-hair theorem}
	Les trous noirs sont totalement définis par un maximum de 3 paramètres. La masse, la charge et le moment angulaire.	$$M\,, Q\,, J $$
\end{block}

\begin{block}{L'insignifiance de la charge}
\begin{itemize}
				  \setlength\itemsep{0em}
	\item les forces électromagnétiques sont bien plus importantes que les forces gravitationnelles
\item le trou noir chargé va capturer toutes les particules de charge contraire disponibles et se neutraliser presque entièrement
\item la charge électrique des trous noirs peut être ignorée
\end{itemize}
	\end{block}



\vspace{10pt}
\pause
\begin{center}
	{\Large Les Trous Noirs Astrophysiques Sont Des Trous Noirs De Kerr}
\end{center}

\end{frame}