\section{Les trous noirs en astrophysique}

\subsection{Les types de trous noirs}
\begin{frame}{\underline{\secname} : {\small \subsecname }}
  \begin{columns}
  	    \begin{column}{0.7\linewidth}

\begin{itemize}
	 \setlength\itemsep{0em}
	\item Trous noirs stellaires.
	 \begin{itemize}
	 	\item ~ 100m dans notre galaxie
	 	\item une masse entre 5 – 64 \(M_\odot\)
\item  le plus proche: HR 6819 situé \'{a} 1120 $A.L.$ %avec 5.0 \pm 0.4 \(M_\odot\)
	\end{itemize}
	\item Trous noirs intermédiaires.
\begin{itemize}
	\item peut se trouver dans les amas globulaire
	\item  masse entre 100 et 10 000 \(M_\odot\)
\end{itemize}

	\item Trous noirs supermassifs.
	 \begin{itemize}
\item  Le plus grand: TON618 avec  66 milliards \(M_\odot\)
\item Les vrais monstres !
\item au centre de toutes les galaxies spirales et elliptiques

	\end{itemize}

%	\item Trous noirs primordiaux. 
%	 \begin{itemize}
%		\item seulement théorique, jamais observé
%		\item Peut avoir été créé peu après le Big Bang
%		\item Peut être créé dans les accélérateurs de particules
%	\end{itemize}
\end{itemize} 

 \end{column}
\begin{column}{0.3\linewidth}
      \begin{figure}
	\centering
\includegraphics[width=\linewidth]{figures/Black_hole_-_Messier_87_crop_max_res.jpg}
	\caption{Image du trou noir supermassif M87*}
\end{figure}
\end{column}
\end{columns}

\end{frame}


{
%	\setbeamercolor{background canvas}{bg=black}

\subsection{La formation des Trous Noirs}
\begin{frame}{\underline{\secname} : {\small \subsecname}}

%\usebackgroundtemplate{\includegraphics[width=\paperwidth]{figures/formation.png}}


\begin{tikzpicture}
\node[anchor=south west,inner sep=0] (image) at (0,0) {\includegraphics[width=0.9\paperwidth,height=0.7\paperheight]{figures/use5128.png}};
\framenode[15pt]{image} % opt. arg. is fade radius; mand. arg. is node name to frame
\end{tikzpicture}
%\includegraphics[width=\paperwidth,height=\paperheight]{figures/formation.png}


\begin{itemize}
		 \setlength\itemsep{0em}
\item 1.4 \(M_\odot\)  limite de Chandrasekhar : Max d’une naine blanche

\item 3 \(M_\odot\) limite de Tolman-Oppenheimer-Volkoff : Max d’une étoile à neutron
\end{itemize}




\end{frame}

}

\begin{frame}{\underline{\secname} : {\small \subsecname}}

\begin{center}
	\textbf{La formation des Trous Noirs intermédiaires et supermassifs}
\end{center}
 \begin{columns}
	
	\begin{column}{0.5\linewidth}
				      \begin{figure}
			\includegraphics[width=\linewidth,height=140pt]{figures/une-trois-trous-noirs-collision-1024x535}
			\caption{Collisions de trous noirs}
			
			
		\end{figure}
	\end{column}
	\begin{column}{0.5\linewidth}
		      \begin{figure}
	\includegraphics[width=\linewidth,height=140pt]{figures/bhbinary}
	\caption{Accrétion de la matière}
	
	
\end{figure}
	\end{column}
\end{columns}
\end{frame}

\subsection{La détection des trous noirs}
\begin{frame}{\underline{\secname} :Comment détecter un trou noir ?}

\begin{center}
	\textbf{Comment détecter un trou noir ?}
\end{center}

 \begin{columns}

	\begin{column}{0.6\linewidth}	
	\begin{itemize}  \setlength\itemsep{0.5em}
		\item  Les systèmes binaires
%		\item  Les rayons gamma
%
%\begin{itemize}
%	\item 			 L’explosion d’une étoile massive en supernova
%\end{itemize}
%
%		\item  La présence de jets
%
%\begin{itemize}
%	\item 			Disque d’accrétion $\rightarrow$ Rayonnement $X$
%\end{itemize}
%
%		\item  La détection des ondes gravitationnelles


	\item 		détecteurs interférométriques d’ondes gravitationnelles 	(LIGO $\&$ Virgo)
	\item 			Disque d’accrétion $\rightarrow$ Rayonnement $X$

	\end{itemize}	
		
	\end{column}
	\begin{column}{0.4\linewidth}
		
		\begin{block}
	
 %	\includegraphics[width=\paperwidth,height=\paperheight]{figures/bhbinary_xmm_960.jpg}
	
	% \includegraphics[width=\paperwidth,height=\paperheight]{figures/jets.png}
\begin{center}

		      \begin{figure}
		      	\includegraphics[width=\linewidth,height=120pt]{figures/ligo}
		   	\caption{le détecteur LIGO}

		
	\end{figure}
\end{center}
	
	
			
		\end{block}	
	\end{column}
\end{columns}
 
\end{frame}